%Preamble

\documentclass[12pt]{article}
\usepackage{amssymb}
\usepackage{amsfonts}
\usepackage{amsmath}
\usepackage[nohead]{geometry}
\usepackage{setspace}
\usepackage[bottom, hang, flushmargin]{footmisc}
\usepackage{indentfirst}
\usepackage{endnotes}
\usepackage{graphicx}
\usepackage{rotating}
\usepackage{natbib}
\usepackage{enumerate}
\usepackage{hyperref}
\setcounter{MaxMatrixCols}{30}
\newtheorem{theorem}{Theorem}
\newtheorem{acknowledgement}{Acknowledgement}
\newtheorem{algorithm}[theorem]{Algorithm}
\newtheorem{axiom}[theorem]{Axiom}
\newtheorem{case}[theorem]{Case}
\newtheorem{claim}[theorem]{Claim}
\newtheorem{conclusion}[theorem]{Conclusion}
\newtheorem{condition}[theorem]{Condition}
\newtheorem{conjecture}[theorem]{Conjecture}
\newtheorem{corollary}[theorem]{Corollary}
\newtheorem{criterion}[theorem]{Criterion}
\newtheorem{definition}[theorem]{Definition}
\newtheorem{example}[theorem]{Example}
\newtheorem{exercise}[theorem]{Exercise}
\newtheorem{lemma}[theorem]{Lemma}
\newtheorem{notation}[theorem]{Notation}
\newtheorem{problem}[theorem]{Problem}
\newtheorem{proposition}{Proposition}
\newtheorem{remark}[theorem]{Remark}
\newtheorem{solution}[theorem]{Solution}
\newtheorem{summary}[theorem]{Summary}
\newenvironment{proof}[1][Proof]{\noindent\textbf{#1.} }{\ \rule{0.5em}{0.5em}}
\geometry{left=1in,right=1in,top=1.00in,bottom=1.0in}

\begin{document}

\singlespacing

\noindent {EC 380: International Economic Issues \\ Instructor: P. Economides \\ Problem Set 3 \\ Fall 2022 \\ Due: 11:59 p.m. on Friday, October 28th}

\bigskip

\doublespacing



\section*{Setup}

\noindent 
Answers must be submitted online through Canvas by the stated deadline (see above).
Please prioritize posting your submission in PDF format.

\section*{Questions}

\noindent Q1. Answer the following short questions:

\begin{enumerate}[1)]
	
	\item Why is the effect of tariffs on large countries' welfare ambiguous when compared to small country outcomes?
	
	\vspace{1in}
	
	\item What does recent research on the multifibre arrangement suggest about its affect on trade outcomes for the US? Consider both price and quantity changes.  
	
	\vspace{1in}
	
	\newpage
	
	\item Why do producers prefer quotas over tariffs in a small country setting? Use visualization to support your argument. 
	
	\vspace{3in}
	
	\item Do you consider an EU ban on GMO crop imports an act of protectionism? Justify your answer.  
	
	\vspace{1.5in}
	
	\item Describe at least two flaws with use of the protection rate measure for evaluating the degree of protection a country applies to a given good. 
	
	\vspace{1in}
	
\end{enumerate}

\newpage

\noindent Q2. Suppose we are in an autarky scenario and considering the market for an imported good at Home. 
Use the following demand and supply functions for solving the various equilibrium scenarios.
\begin{align*}
\text{Demand:} \ \  P =& 120 - \frac{4}{7} Q_d\\
\text{Supply:}\ \  P =& \frac{1}{4} Q_s
\end{align*}

\begin{enumerate}[1)]
	\item Consider the autarky scenario first. Sketch the supply and demand curves, with the appropriate labeling for the equilibrium point, surplus regions.
	
	\vspace{3in}
	
	\item Report the coordinates of the equilibrium point, which represent the price and quantity the market operates at.
	
	\vspace{1in}
	
	
	\newpage
	
	\item Calculate the consumer and producer surplus values under autarky. What is total welfare for the economy?
	
	\vspace{1.4in}
	
	\noindent Suppose Home opens up to free-trade, and becomes exposed to a world price, $P_w = 25$. 
	
	\item Re-sketch the market with the new price line and corresponding equilibria points for quantity demanded and supplied. 
	Calculate the equilibrium values for quantities, imports and surplus values. 
	Highlight the change in welfare, relative to autarky. 
	
	\vspace{4in}
	
	\newpage
	
	\noindent Consider a case in which the government intervenes, setting a tariff rate of $t=20$. 
	
	\item Sketch the updated demand \& supply schedule. Upon appropriate labeling the diagram, highlight which regions are the efficiency and deadweight loss areas, respectively.
	
	\vspace{3.5in}
	
	\item Calculate the equilibria for quantity supplied, quantity demanded, imports and surpluses (consumer, producer, government).
	Find the change in welfare, relative to free-trade. 
	
	\vspace{2in}
	
\end{enumerate}

\newpage


\noindent Q3. Suppose we are in an autarky scenario and considering the market for an imported good at Home. 
Use the following demand and supply functions for solving the various equilibrium scenarios.
\begin{align*}
\text{Demand:} \ \  P = & 235 - \frac{2}{3} Q_d\\
\text{Supply:}\ \  P = & 50  + 4 Q_s
\end{align*}

\begin{enumerate}[1)]
	\item Consider the autarky scenario first. Sketch the supply and demand curves, with the appropriate labeling for the equilibrium point, surplus regions.
	
	\vspace{3in}
	
	\item Report the coordinates of the equilibrium point, which represent the price and quantity the market operates at.
	
	\vspace{1in}
	
	
	\newpage
	
	\item Calculate the consumer and producer surplus values under autarky.  What is total welfare for the economy?
	
	\vspace{1.4in}
	
	\noindent Suppose Home opens up to free-trade, and becomes exposed to a world price, $P_w = 140$. 
	
	\item Re-sketch the market with the new price line and corresponding equilibria points for quantity demanded and supplied. 
	Calculate the equilibrium values for quantities, imports and surplus values. 
	Highlight the change in welfare, relative to autarky. 
	
	\vspace{4in}
	
	\newpage
	
	\noindent Consider a case in which the government intervenes, setting a tariff rate of $t=20$. 
	
	\item Sketch the updated demand \& supply schedule. Upon appropriate labeling the diagram, highlight which region that represents government revenue.
	
	\vspace{3.5in}
	
	\item Calculate the equilibria for quantity supplied, quantity demanded, imports and surpluses (consumer, producer, government). 
	Find the change in welfare, relative to free-trade. 
	
	\vspace{2in}
	
\end{enumerate}

\newpage

\noindent Q4. Tariffs on final and input goods are 25\% and 12\%, respectively. 

\begin{table}[!h]
	\centering
	\begin{tabular}[t]{l c c c}
		\hline
		&&&\\
		Variable & No Tariff & + Tariff on Final Good & + Tariff on Input Good \\
		&&&\\
		\hline
		&&&\\
		Price of Domestic Final Good & 2220 & & \\
		Value of Imported Inputs & 670 & & \\
		Domestic Value-Added &	1550	&&\\
				&&&\\
		Effective Rate of Protection, \% &	0	&& \\
		&&&\\
		\hline
	\end{tabular}
\end{table}



\noindent Complete the entries above and express the effective rate of protection in each case. 
Display your workings in the space provided below.



\end{document}