%Preamble

\documentclass[12pt]{article}
\usepackage{amssymb}
\usepackage{amsfonts}
\usepackage{amsmath}
\usepackage[nohead]{geometry}
\usepackage{setspace}
\usepackage[bottom, hang, flushmargin]{footmisc}
\usepackage{indentfirst}
\usepackage{endnotes}
\usepackage{graphicx}
\usepackage{rotating}
\usepackage{natbib}
\usepackage{enumerate}
\usepackage{hyperref}
\setcounter{MaxMatrixCols}{30}
\newtheorem{theorem}{Theorem}
\newtheorem{acknowledgement}{Acknowledgement}
\newtheorem{algorithm}[theorem]{Algorithm}
\newtheorem{axiom}[theorem]{Axiom}
\newtheorem{case}[theorem]{Case}
\newtheorem{claim}[theorem]{Claim}
\newtheorem{conclusion}[theorem]{Conclusion}
\newtheorem{condition}[theorem]{Condition}
\newtheorem{conjecture}[theorem]{Conjecture}
\newtheorem{corollary}[theorem]{Corollary}
\newtheorem{criterion}[theorem]{Criterion}
\newtheorem{definition}[theorem]{Definition}
\newtheorem{example}[theorem]{Example}
\newtheorem{exercise}[theorem]{Exercise}
\newtheorem{lemma}[theorem]{Lemma}
\newtheorem{notation}[theorem]{Notation}
\newtheorem{problem}[theorem]{Problem}
\newtheorem{proposition}{Proposition}
\newtheorem{remark}[theorem]{Remark}
\newtheorem{solution}[theorem]{Solution}
\newtheorem{summary}[theorem]{Summary}
\newenvironment{proof}[1][Proof]{\noindent\textbf{#1.} }{\ \rule{0.5em}{0.5em}}
\geometry{left=1in,right=1in,top=1.00in,bottom=1.0in}

\begin{document}

\singlespacing

\noindent {EC 380: International Economic Issues \\ Instructor: P. Economides \\ Practice Problems: Final Exam \\ Fall 2022}

\bigskip

\doublespacing



\section*{Setup}

\noindent 
The solutions for the following questions are available on Canvas under `practice-sol.pdf'.
The details surrounding reaching each of these answers can be discussed during email/office hour correspondence.

\subsection*{Tariff Theory}

\noindent Q1. Suppose we are in an autarky scenario and considering the market for an imported good at Home. 
Use the following demand and supply functions for solving the various equilibrium scenarios.
\begin{align*}
\text{Demand:} \ \  P = & 315 - \frac{3}{7} Q_d\\
\text{Supply:}\ \  P = & 50  + \frac{1}{3} Q_s
\end{align*}

\begin{enumerate}[1)]
	\item Consider the autarky scenario first. Sketch the supply and demand curves, with the appropriate labeling for the equilibrium point, surplus regions.
	
	\vspace{2.4in}
	
	\item Report the coordinates of the equilibrium point, which represent the price and quantity the market operates at.
	
	\vspace{1in}
	
	
	\item Calculate the consumer and producer surplus values under autarky.  What is total welfare for the economy?
	
	\vspace{1.4in}
	
	\noindent Suppose Home opens up to free-trade, and becomes exposed to a world price, $P_w = 120$. 
	
	\item Re-sketch the market with the new price line and corresponding equilibria points for quantity demanded and supplied. 
	Calculate the equilibrium values for quantities, imports and surplus values. 
	Highlight the change in welfare, relative to autarky. 
	
	\vspace{4in}
	
	\newpage
	
	\noindent Consider a case in which the government intervenes, setting a tariff rate of $t=28$. 
	
	\item Sketch the updated demand \& supply schedule. Upon appropriate labeling the diagram, highlight which region that represents government revenue.
	
	\vspace{3.5in}
	
	\item Calculate the equilibria for quantity supplied, quantity demanded, imports and surpluses (consumer, producer, government). 
	Find the change in welfare, relative to free-trade. 
	
	\vspace{2in}
	
	\item What are the respective efficiency loss and deadweight loss amounts associated with this form of government intervention?
	
\end{enumerate}

\newpage

\noindent Q2. Suppose we are in an autarky scenario and considering the market for an imported good at Home. 
Use the following demand and supply functions for solving the various equilibrium scenarios.
\begin{align*}
\text{Demand:} \ \  P = & 275 - \frac{1}{2} Q_d\\
\text{Supply:}\ \  P = & 30  + \frac{5}{8} Q_s
\end{align*}

\begin{enumerate}[1)]
	\item Consider the autarky scenario first. Sketch the supply and demand curves, with the appropriate labeling for the equilibrium point, surplus regions.
	
	\vspace{2.4in}
	
	\item Report the coordinates of the equilibrium point, which represent the price and quantity the market operates at.
	
	\vspace{1in}
	
	\item Calculate the consumer and producer surplus values under autarky.  What is total welfare for the economy?
	
	\vspace{1.4in}
	
	\noindent Suppose Home opens up to free-trade, and becomes exposed to a world price, $P_w = 85$. 
	
	\item Re-sketch the market with the new price line and corresponding equilibria points for quantity demanded and supplied. 
	Calculate the equilibrium values for quantities, imports and surplus values. 
	Highlight the change in welfare, relative to autarky. 
	
	\vspace{3in}

	
	\noindent Consider a case in which the government intervenes, setting a tariff rate of $t=25$. 
	
	\item Sketch the updated demand \& supply schedule. Upon appropriate labeling the diagram, highlight which region that represents government revenue.
	
	\vspace{3.5in}
	
	\item Calculate the equilibria for quantity supplied, quantity demanded, imports and surpluses (consumer, producer, government). 
	Find the change in welfare, relative to free-trade. 
	
	\vspace{2in}
	
	\item What are the respective efficiency loss and deadweight loss amounts associated with this form of government intervention?
	
\end{enumerate}

\newpage

\noindent Q3. Suppose we are in an autarky scenario and considering the market for an imported good at Home. 
Use the following demand and supply functions for solving the various equilibrium scenarios.
\begin{align*}
\text{Demand:} \ \  P = & 550 - 2 Q_d\\
\text{Supply:}\ \  P = & 100  + 2.5 Q_s
\end{align*}

\begin{enumerate}[1)]
	\item Consider the autarky scenario first. Sketch the supply and demand curves, with the appropriate labeling for the equilibrium point, surplus regions.
	
	\vspace{2.4in}
	
	\item Report the coordinates of the equilibrium point, which represent the price and quantity the market operates at.
	
	\vspace{1in}
	
	\item Calculate the consumer and producer surplus values under autarky.  What is total welfare for the economy?
	
	\vspace{1.4in}
	
	\noindent Suppose Home opens up to free-trade, and becomes exposed to a world price, $P_w = 190$. 
	
	\item Re-sketch the market with the new price line and corresponding equilibria points for quantity demanded and supplied. 
	Calculate the equilibrium values for quantities, imports and surplus values. 
	Highlight the change in welfare, relative to autarky. 
	
	\vspace{3in}
	
	\noindent Consider a case in which the government intervenes, setting a tariff rate of $t=45$. 
	
	\item Sketch the updated demand \& supply schedule. Upon appropriate labeling the diagram, highlight which region that represents government revenue.
	
	\vspace{3.5in}
	
	\item Calculate the equilibria for quantity supplied, quantity demanded, imports and surpluses (consumer, producer, government). 
	Find the change in welfare, relative to free-trade. 
	
	\vspace{2in}
	
	\item What are the respective efficiency loss and deadweight loss amounts associated with this form of government intervention?
	
\end{enumerate}

\newpage

\noindent Q4. Suppose we are in an autarky scenario and considering the market for an imported good at Home. 
Use the following demand and supply functions for solving the various equilibrium scenarios.
\begin{align*}
\text{Demand:} \ \  P = & 150 - \frac{8}{9} Q_d\\
\text{Supply:}\ \  P = & 5  + \frac{7}{9} Q_s
\end{align*}

\begin{enumerate}[1)]
	\item Consider the autarky scenario first. Sketch the supply and demand curves, with the appropriate labeling for the equilibrium point, surplus regions.
	
	\vspace{2.4in}
	
	\item Report the coordinates of the equilibrium point, which represent the price and quantity the market operates at.
	
	\vspace{1in}
	
	\item Calculate the consumer and producer surplus values under autarky.  What is total welfare for the economy?
	
	\vspace{1.4in}
	
	\noindent Suppose Home opens up to free-trade, and becomes exposed to a world price, $P_w = 35$. 
	
	\item Re-sketch the market with the new price line and corresponding equilibria points for quantity demanded and supplied. 
	Calculate the equilibrium values for quantities, imports and surplus values. 
	Highlight the change in welfare, relative to autarky. 
	
	\vspace{3in}
	
	\noindent Consider a case in which the government intervenes, setting a tariff rate of $t=15$. 
	
	\item Sketch the updated demand \& supply schedule. Upon appropriate labeling the diagram, highlight which region that represents government revenue.
	
	\vspace{3.5in}
	
	\item Calculate the equilibria for quantity supplied, quantity demanded, imports and surpluses (consumer, producer, government). 
	Find the change in welfare, relative to free-trade. 
	
	\vspace{2in}

	\item What are the respective efficiency loss and deadweight loss amounts associated with this form of government intervention?
	
\end{enumerate}

\newpage


\noindent Consider the following breakdowns of domestic value-added and imported input contributions to final output of local firms in an economy.
Complete the entries above and express the effective rate of protection in each case.

\noindent Q5. Consider a final good tariff of 12 percent and an input good tariff of 37 percent.

\begin{table}[!h]
	\centering
	\begin{tabular}[t]{l c c c}
		\hline
		&&&\\
		Variable & No Tariff & + Tariff on Final Good & + Tariff on Input Good \\
		&&&\\
		\hline
		&&&\\
		Price of Domestic Final Good & 2870 & & \\
		Value of Imported Inputs & 870 & & \\
		Domestic Value-Added &	2000	&&\\
		&&&\\
		Effective Rate of Protection, \% &	0	&& \\
		&&&\\
		\hline
	\end{tabular}
\end{table}

\noindent Q6. Consider a final good tariff of 17 percent and an input good tariff of 20 percent.

\begin{table}[!h]
	\centering
	\begin{tabular}[t]{l c c c}
		\hline
		&&&\\
		Variable & No Tariff & + Tariff on Final Good & + Tariff on Input Good \\
		&&&\\
		\hline
		&&&\\
		Price of Domestic Final Good & 1910 & & \\
		Value of Imported Inputs & 430 & & \\
		Domestic Value-Added &	1480	&&\\
		&&&\\
		Effective Rate of Protection, \% &	0	&& \\
		&&&\\
		\hline
	\end{tabular}
\end{table}

\noindent Q7.Consider a final good tariff of 10 percent and an input good tariff of 5 percent.
 
\begin{table}[!h]
	\centering
	\begin{tabular}[t]{l c c c}
		\hline
		&&&\\
		Variable & No Tariff & + Tariff on Final Good & + Tariff on Input Good \\
		&&&\\
		\hline
		&&&\\
		Price of Domestic Final Good & 4830 & & \\
		Value of Imported Inputs & 2415 & & \\
		Domestic Value-Added &	2415	&&\\
		&&&\\
		Effective Rate of Protection, \% &	0	&& \\
		&&&\\
		\hline
	\end{tabular}
\end{table}


\newpage

\subsection*{Balance of Payments}

\noindent Q8. Consider the following balance of payments for a given country.

\begin{table}[!h]
	\centering
	\begin{tabular}[t]{l l c }
		\hline
		&&\\
		ID & Description & Billions, USD \\
		&&\\
		\hline
		&&\\
		1. & Current Account Balance & -550\\
		2. & Capital Account Balance & 100\\
		3. & Financial Account	& -\\
		3a. & Net acq of financial assets, excl financial der (increase/outflow (+))& 870\\
		3b. & Net inc of liabilities, excl financial der (increase/inflow (+))	& 1200 \\
		3c. & Net change in financial derivatives & -80\\
		4. & Statistical Discrepancy & \\
		5. & Memoranda & \\
		5a. & Balance on current and capital accounts & \\
		5b. & Balance on financial account & \\
		\hline
	\end{tabular}
\end{table}

\begin{enumerate}[1)]
	\item In theory, what should the difference between items (5a.) and (5b.) be?
	
	\vspace{0.4in}
	
	\item Report the associated value of item (5a). Show your workings.
	
	\vspace{0.4in}
	%$$-550+100=-450$$
	\vspace{0.2in}
	
	\item Report the associated value of item (5b). Show your workings.
	
	\vspace{0.4in}
	%$$870-(1200)-80=-410$$
	\vspace{0.2in}
	
	\item What is the measure of statistical discrepancy measured as in this case (4.)?
	
	\vspace{0.4in}
	%$$-450+410=-40$$
	\vspace{0.2in}
	
	\item Would this country be considered a case of CA surplus or CA deficit?
	
	\vspace{0.4in}	
	%Since the current account balance is negative, this would be a CA deficit case.
	\vspace{0.2in} 
	
\end{enumerate}

\newpage

\noindent Q9. Consider the following balance of payments for a given country.

\begin{table}[!h]
	\centering
	\begin{tabular}[t]{l l c }
		\hline
		&&\\
		ID & Description & Billions, USD \\
		&&\\
		\hline
		&&\\
		1. & Current Account Balance & -1830\\
		2. & Capital Account Balance & 290\\
		3. & Financial Account	& -\\
		3a. & Net acq of financial assets, excl financial der (increase/outflow (+))& 7000\\
		3b. & Net inc of liabilities, excl financial der (increase/inflow (+))	& -6300 \\
		3c. & Net change in financial derivatives & -14780\\
		4. & Statistical Discrepancy & \\
		5. & Memoranda & \\
		5a. & Balance on current and capital accounts & \\
		5b. & Balance on financial account & \\
		\hline
	\end{tabular}
\end{table}

\begin{enumerate}[1)]
	\item In theory, what should the difference between items (5a.) and (5b.) be?
	
	\vspace{0.4in}
	
	\item Report the associated value of item (5a). Show your workings.
	
	\vspace{0.4in}
	%$$-1830+290=-1540$$
	\vspace{0.2in}
	
	\item Report the associated value of item (5b). Show your workings.
	
	\vspace{0.4in}
	%$$7000-(-6300)-14780=-1480$$
	\vspace{0.2in}
	
	\item What is the measure of statistical discrepancy measured as in this case (4.)?
	
	\vspace{0.4in}
	%$$-1540+1480=-60$$
	\vspace{0.2in}
	
	\item Would this country be considered a case of CA surplus or CA deficit?
	
	\vspace{0.4in}	
	%Since the current account balance is negative, this would be a CA deficit case.
	\vspace{0.2in} 
	
\end{enumerate}


\subsection*{Exchange Rates}

\noindent Q10. Consider the following demand and the supply curves of foreign currency, where $\text{ExR}$ represents the local currency to foreign currency (e.g. USD-GBP) exchange rate.
$\text{FC}$ represents the units of foreign currency reserves held in the ``local" economy. 
\begin{align*}
D: \text{ExR} =  57 - 0.061 \text{FC} \ \ , & \ \ S: \text{ExR} =  2 + 0.012 \text{FC}
\end{align*}

\begin{enumerate}[1)]

\item What is the market clearing rate of exchange and the associated level of foreign currency reserves?

% 1.38                 35
\vspace{0.8in}

\item Consider a case in which the market anticipates a new technology being released abroad, which causes demand for foreign currency reserves to rise by 2 units, such that the new demand curve can be represented by $D^{'} = D + 2$.
What are the new exchange rate and currency reserve values?

\vspace{0.8in}
%  1.5                 40

\item How would you describe the change in both currencies? Which has depreciated and which has appreciated?

% Local depreciated, foreign appreciated
\vspace{0.8in}

\item What is the percentage change in currency reserves?

% 100*5/35 = 14.3%
\vspace{0.8in}

\end{enumerate}

\newpage

\noindent Q11. Consider the following demand and the supply curves of foreign currency, where $\text{ExR}$ represents the local currency to foreign currency (e.g. USD-GBP) exchange rate.
$\text{FC}$ represents the units of foreign currency reserves held in the ``local" economy. 
\begin{align*}
D: \text{ExR} =  180 - 0.290 \text{FC} \ \ , & \ \ S: \text{ExR} =  50 + 0.251 \text{FC}
\end{align*}

\begin{enumerate}[1)]
	
\item What is the market clearing rate of exchange and the associated level of foreign currency reserves?

% 1.38                 35
\vspace{0.8in}

\item Consider a case in which a large cache of foreign currency is discovered following an audit of local steel factories, causing foreign currency reserves in increase in supply by 40 units. This increase in supply is represented on the new supply curve by $S^{'} = S - 40$.\footnote{Shifting the supply curve to the right is equivalent to reducing each point on the line down by a constant amount of units.}
What are the new exchange rate and currency reserve values?

\vspace{0.8in}
%  1.5                 40

\item How would you describe the change in both currencies? Which has depreciated and which has appreciated?

% Local depreciated, foreign appreciated
\vspace{0.8in}

\item What is the percentage change in currency reserves?

% 100*5/35 = 14.3%
\vspace{0.8in}
	
\end{enumerate}

\noindent Q12. Consider the following demand and the supply curves of foreign currency, where $\text{ExR}$ represents the local currency to foreign currency (e.g. USD-GBP) exchange rate.
$\text{FC}$ represents the units of foreign currency reserves held in the ``local" economy. 
\begin{align*}
D: \text{ExR} =  1080 - 45 \text{FC} \ \ , & \ \ S: \text{ExR} =  200 + 22 \text{FC}
\end{align*}

\begin{enumerate}[1)]
	
\item What is the market clearing rate of exchange and the associated level of foreign currency reserves?

% 1.38                 35
\vspace{1in}

\item Consider a case in which the domestic and foreign interest rates, $i \& i^*$, both rise such that foreign currency demand sees a 10 level-shift increase and supply experiences a level shift of 80 units. The new curves are defined as $D^{'} = D + 10 \ , \ S^{'} = S - 80$.
What are the new exchange rate and currency reserve values?

\vspace{1in}
%  1.5                 40

\item How would you describe the change in both currencies? Which has depreciated and which has appreciated?

% Local depreciated, foreign appreciated
\vspace{1in}

\item What is the percentage change in currency reserves?

% 100*5/35 = 14.3%
\vspace{1in}
	
\end{enumerate}

\end{document}