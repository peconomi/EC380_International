%Preamble

\documentclass[12pt]{article}
\usepackage{amssymb}
\usepackage{amsfonts}
\usepackage{amsmath}
\usepackage[nohead]{geometry}
\usepackage{setspace}
\usepackage[bottom, hang, flushmargin]{footmisc}
\usepackage{indentfirst}
\usepackage{endnotes}
\usepackage{graphicx}
\usepackage{rotating}
\usepackage{natbib}
\usepackage{enumerate}
\usepackage{hyperref}
\setcounter{MaxMatrixCols}{30}
\newtheorem{theorem}{Theorem}
\newtheorem{acknowledgement}{Acknowledgement}
\newtheorem{algorithm}[theorem]{Algorithm}
\newtheorem{axiom}[theorem]{Axiom}
\newtheorem{case}[theorem]{Case}
\newtheorem{claim}[theorem]{Claim}
\newtheorem{conclusion}[theorem]{Conclusion}
\newtheorem{condition}[theorem]{Condition}
\newtheorem{conjecture}[theorem]{Conjecture}
\newtheorem{corollary}[theorem]{Corollary}
\newtheorem{criterion}[theorem]{Criterion}
\newtheorem{definition}[theorem]{Definition}
\newtheorem{example}[theorem]{Example}
\newtheorem{exercise}[theorem]{Exercise}
\newtheorem{lemma}[theorem]{Lemma}
\newtheorem{notation}[theorem]{Notation}
\newtheorem{problem}[theorem]{Problem}
\newtheorem{proposition}{Proposition}
\newtheorem{remark}[theorem]{Remark}
\newtheorem{solution}[theorem]{Solution}
\newtheorem{summary}[theorem]{Summary}
\newenvironment{proof}[1][Proof]{\noindent\textbf{#1.} }{\ \rule{0.5em}{0.5em}}
\geometry{left=1in,right=1in,top=1.00in,bottom=1.0in}

\begin{document}

\singlespacing

\noindent {EC 380: International Economic Issues \\ Instructor: P. Economides \\ Problem Set 1 \\ Fall 2022 \\ Due: 11:59 p.m. on Friday, October 7th}

\bigskip

\doublespacing



\section*{Setup}

\noindent 
Answers must be submitted online through Canvas by the stated deadline (see above).

\section*{Questions}

\noindent Q1. Answer the following short questions:

\begin{enumerate}[1)]
	
	\item How is the Trade to GDP ratio measured? How has this value been trending over the years?
	
	\vspace{1in}
	
	\item How does labor mobility far compared to the early 1900s?
	
	\vspace{1in}
	
	\item Describe two features of contemporary international economic relations.
	
	\vspace{1in}
	
	\newpage
	
	\item Briefly describe what a trade deficit is and the US track record on deficits. 
	
	\vspace{1in}
	
	\item Describe the employment possibilities and occupations open to students of international economics.
	
	\vspace{1in}
	
\end{enumerate}

\noindent Q2. Suppose we are considering a Ricardian model setting, where countries have not yet opened up to trade.
Two goods are produced exclusively by domestic labor supplies, Pots and Plants. 

\noindent Home and foreign maintain the following marginal productivities of labor (MPL) in producing products. The two countries labor pools $\bar{L}=35$ are equal in both countries. 

\begin{table}[!h]
	\centering
	\begin{tabular}[t]{l c c }
		\hline
		&&\\
		MPL & Pots & Plants  \\
		&&\\
		\hline
		&&\\
		Home & 2 & 5  \\
		Foreign & 3 & 7 \\
		&&\\
		\hline
	\end{tabular}
\end{table}



Consider the autarky scenario where countries do not exchange goods. Complete the following questions to obtain the two countries consumption and production equilibria. 

\begin{enumerate}[1)]

\item Which country has comparative advantage in producing Plants?

\newpage

\item What are the max quantities of each good that Home and Foreign can produce?

\vspace{2.0in}

\begin{table}[!h]
	\centering
	\begin{tabular}[t]{l c c }
		\hline
		&&\\
		Max Output & Pots & Plants  \\
		&&\\
		\hline
		&&\\
		Home &  &   \\
		Foreign &  &  \\
		&&\\
		\hline
	\end{tabular}
\end{table}

\item Sketch the PPFs of Home and Foreign in a single graph, given max output levels. Be sure to correctly label the graph for full points. 

\bigskip

\bigskip

\bigskip

\bigskip

\newpage

\item Suppose Home prefers consuming 3 plants for every pot consumed. Calculate the consumption bundle of Home and sketch it on a PPF graph.

\bigskip

\bigskip

\bigskip

\bigskip

\bigskip

\bigskip

\bigskip

\bigskip

\bigskip

\bigskip

\bigskip

\bigskip

\bigskip

\bigskip

\bigskip

\bigskip

\bigskip

\bigskip  


\newpage

\item Suppose Foreign prefers consuming 5 plants for every pot consumed. Calculate the consumption bundle of Foreign and sketch it on a PPF graph. 

\vspace{2in}

\item Consider a shock to the economy where Home suddenly becomes more productive at making pots. Do comparative advantages change for a case in which $MPL^H_{Pots}=8$?
How does Home's production bundle change?

\vspace{4in}

\end{enumerate}

\newpage


\noindent Q3. Suppose we are considering a Ricardian model setting, where countries have not yet opened up to trade.
Two goods are produced exclusively by domestic labor supplies, Air Pumps and Car Tires. 

\noindent Home and foreign maintain the following marginal productivities of labor (MPL) in producing products. The two countries labor pools $\bar{L}=50$ are equal in both countries. 

\begin{table}[!h]
	\centering
	\begin{tabular}[t]{l c c }
		\hline
		&&\\
		MPL & Air Pump & Car Tires  \\
		&&\\
		\hline
		&&\\
		Home & 14 & 8  \\
		Foreign & 9 & 12 \\
		&&\\
		\hline
	\end{tabular}
\end{table}



Consider the autarky scenario where countries do not exchange goods. Complete the following questions to obtain the two countries consumption and production equilibria. 

\begin{enumerate}[1)]
	
	\item Which country has comparative advantage in producing Tires?
	
	\newpage
	
	\item What are the max quantities of each good that Home and Foreign can produce?
	
	\vspace{2.0in}
	
	\begin{table}[!h]
		\centering
		\begin{tabular}[t]{l c c }
			\hline
			&&\\
			Max Output & Air Pump & Car Tires  \\
			&&\\
			\hline
			&&\\
			Home &  &   \\
			Foreign &  &  \\
			&&\\
			\hline
		\end{tabular}
	\end{table}
	
	\item Sketch the PPFs of Home and Foreign in a single graph, given max output levels. Be sure to correctly label the graph for full points. 
	
	\bigskip
	
	\bigskip
	
	\bigskip
	
	\bigskip
	
	\newpage
	
	\item Suppose Home prefers consuming 8 tires for every pump consumed. Calculate the consumption bundle of Home and sketch it on a PPF graph.
	
	\bigskip
	
	\bigskip
	
	\bigskip
	
	\bigskip
	
	\bigskip
	
	\bigskip
	
	\bigskip
	
	\bigskip
	
	\bigskip
	
	\bigskip
	
	\bigskip
	
	\bigskip
	
	\bigskip
	
	\bigskip
	
	\bigskip
	
	\bigskip
	
	\bigskip
	
	\bigskip  
	
	
	\newpage
	
	\item Suppose Foreign prefers consuming 12 tires for every pump consumed. Calculate the consumption bundle of Foreign and sketch it on a PPF graph. 
	
	\vspace{2in}
	
	\item Consider a migrant boom occurs at foreign, where the Home workforce becomes $\bar{L}=60$. Do comparative advantages change for either country?
	How does Home's production bundle change?
	
	\vspace{4in}
	
\end{enumerate}

\end{document}