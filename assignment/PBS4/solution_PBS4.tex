%Preamble

\documentclass[12pt]{article}
\usepackage{amssymb}
\usepackage{amsfonts}
\usepackage{amsmath}
\usepackage[nohead]{geometry}
\usepackage{setspace}
\usepackage[bottom, hang, flushmargin]{footmisc}
\usepackage{indentfirst}
\usepackage{endnotes}
\usepackage{graphicx}
\usepackage{rotating}
\usepackage{natbib}
\usepackage{enumerate}
\usepackage{hyperref}
\setcounter{MaxMatrixCols}{30}
\newtheorem{theorem}{Theorem}
\newtheorem{acknowledgement}{Acknowledgement}
\newtheorem{algorithm}[theorem]{Algorithm}
\newtheorem{axiom}[theorem]{Axiom}
\newtheorem{case}[theorem]{Case}
\newtheorem{claim}[theorem]{Claim}
\newtheorem{conclusion}[theorem]{Conclusion}
\newtheorem{condition}[theorem]{Condition}
\newtheorem{conjecture}[theorem]{Conjecture}
\newtheorem{corollary}[theorem]{Corollary}
\newtheorem{criterion}[theorem]{Criterion}
\newtheorem{definition}[theorem]{Definition}
\newtheorem{example}[theorem]{Example}
\newtheorem{exercise}[theorem]{Exercise}
\newtheorem{lemma}[theorem]{Lemma}
\newtheorem{notation}[theorem]{Notation}
\newtheorem{problem}[theorem]{Problem}
\newtheorem{proposition}{Proposition}
\newtheorem{remark}[theorem]{Remark}
\newtheorem{solution}[theorem]{Solution}
\newtheorem{summary}[theorem]{Summary}
\newenvironment{proof}[1][Proof]{\noindent\textbf{#1.} }{\ \rule{0.5em}{0.5em}}
\geometry{left=1in,right=1in,top=1.00in,bottom=1.0in}

\begin{document}

\singlespacing

\noindent {EC 380: International Economic Issues \\ Instructor: P. Economides \\ Problem Set 4 \\ Fall 2022 \\ Due: 11:59 p.m. on Friday, Nov 25th}

\bigskip

\doublespacing



\section*{Setup}

\noindent 
Answers must be submitted online through Canvas by the stated deadline (see above).
Please prioritize posting your submission in PDF format.

\section*{Questions}

\noindent Q1. Answer the following short questions:

\begin{enumerate}[1)]
	
	\item Describe the current account, list and describe each of its three subcomponents. Describe one of these items in detail. 
	
	\vspace{0.2in}
	The current account balance of payments is a record of a country's international transactions with the rest of the world.
	It is comprised of the trade balance, net primary income and net second income.
	The first element is the value of total exports of goods and services less imports. 
	The second item represents differences in cross-border compensation to employees and the third item represents cross-border transfers of income. 
	\vspace{0.2in}
	
	\item What type of activities does the capital account consist of?
	
	\vspace{0.2in}
	The capital account of the balance of payments is the record of specialized capital transfers. Because it is a measure of transfers and not purchases or sales, it is somewhat similar to the category of secondary income of the current account, but with the major distinction that it applies to capital transfers and not income transfers. Normally, this is a small item and includes relatively infrequent activities such as the transfer of military bases or embassies between countries, debt forgiveness, and the personal assets that migrants carry with them when they cross borders.
	\vspace{0.2in}
	
	\item Regarding the financial account, what subcomponents does this measure include?
	
	\vspace{0.2in}
	Tracks capital ows between economies, usually long-
	lasting items relative to goods or services. Includes  the net acquisition of financial asset, the net occurrence of liabilities, and net flows of financial derivatives.
	\vspace{0.2in}
	
	\item How do the current, capital and financial accounts relate to one another when it comes to their numeric values?  
	
	\vspace{0.2in}
	The financial account, in theory, is equal to the sum of the current and capital accounts. In practice, the difference is represented as statistical discrepancy.
	\vspace{0.2in}
	
	\item Explain the reason why the balance of payments features a statistical discrepancy value.
	
	\vspace{0.2in}
	Impossible to record all transactions and to ensure they are accurately
	measured.
	Amount of net lending or borrowing on the current and capital accounts
	rarely matches the amount implied by the financial account balance.
	The statistical discrepancy is size of this measurement error.
	\vspace{0.2in}
	
\end{enumerate}

\newpage

\noindent Q2. Consider the following balance of payments for a given country.

\begin{table}[!h]
	\centering
	\begin{tabular}[t]{l l c }
		\hline
		&&\\
		ID & Description & Billions, USD \\
		&&\\
		\hline
		&&\\
		1. & Current Account Balance & 1000\\
		2. & Capital Account Balance & 200\\
		3. & Financial Account	& -\\
		3a. & Net acq of financial assets, excl financial der (increase/outflow (+))& 470\\
		3b. & Net inc of liabilities, excl financial der (increase/inflow (+))	& -890 \\
		3c. & Net change in financial derivatives & -200\\
		4. & Statistical Discrepancy & \\
		5. & Memoranda & \\
		5a. & Balance on current and capital accounts & \\
		5b. & Balance on financial account & \\
		\hline
	\end{tabular}
\end{table}




\begin{enumerate}[1)]
	\item In theory, what should the difference between items (5a.) and (5b.) be?
	
	\vspace{0.2in}
	Zero.
	\vspace{0.2in}
	
	\item Report the associated value of item (5a). Show your workings.
	
	\vspace{0.2in}
	$$1000+200=1200$$
	\vspace{0.2in}
	
	\item Report the associated value of item (5b). Show your workings.
	
	\vspace{0.2in}
	$$470-(-890)-200=1160$$
	\vspace{0.2in}
	
	\item What is the measure of statistical discrepancy measured as in this case (4.)?
	
	\vspace{0.2in}
	$$1200-1160=40$$
	\vspace{0.2in}
	
	\item Would this country be considered a case of CA surplus or CA deficit?

	\vspace{0.2in}	
	Since the current account balance is positive, this would be a CA surplus case.
	\vspace{0.2in} 
	
\end{enumerate}


\noindent Q3. Consider a case in which a given economy reports a GNP level of 5.4bn USD. 
The primary income net flows are work 1.2bn, while secondary income transfers are worth 0.2bn. 

\begin{enumerate}[1)]
	\item What is the implied level of GDP under these circumstances? Note that primary income is associated with income flows for compensating employees whereas secondary income is associated with transfers of income. 
	
	\vspace{0.2in}
	GDP= 5.4bn - 1.2bn - 0.2bn
	GDP= 4bn
	\vspace{0.2in}
	
	\item Demonstrate how the current account surplus is present in the measure of GNP, starting from the equation listed below.
	\begin{align*}
	GNP = & GDP + \text{Net Primary Income} + \text{Net Secondary Income}\\
	CA= & NX+ \text{Net Primary Income} + \text{Net Secondary Income}\\
	GNP= & C+I+G+X-M+\text{Net Primary Income} + \text{Net Secondary Income}\\
	GNP= & C+I+G+CA
	\end{align*}

	
	Using GDP, consider the fact that consumption (C) is equal to 1.5bn, investment (I) is 1.8bn and the government runs a balanced budget and collects 0.4bn in revenue (G). 
	
	\item What is the value of {\bf net} exports? Show your work.
	% 1.5 + 1.8 + 0.4 + NX = 4 = 1.5 + S + 0.4
	% C + I + G + X - M  = C + S + T = 4
	% C = 1.5, I = 1.8, G = 0.4, X=1, M=0.7
	% C = 1.5, S = 2.1, T = 0.4   
	\begin{align*}
	4 (GDP) = & 1.5 + 1.8 + 0.4 + NX\\
	\implies NX = & 0.3
	\end{align*}
	
	\item What is the value of savings? Show your work.
	\begin{align*}
	S+T-G= &I+CA\\
	S+0.4-0.4= & 1.8+(1.2+0.2+0.3)\\
	S= &3.5
	\end{align*}
	
	\item Suppose there is a shock to the economy, and the government is forced to run a major budget deficit 1.2bn.
	What does this imply about the tax rate for the country?
	
	\vspace{0.2in}
	This implies taxes were not set high enough to maintain a balanced budget. 
	\vspace{0.2in}
	
	\item Update your measure of GNP to reflect this change in the government budget balance. Show your workings. What is the percentage change in GNP?
	\begin{align*}
	GNP=& C+I+G+CA\\
	GNP=& 1.5+1.8+1.6+(0.3+1.2+0.2)\\
	GNP=& 6.6\\
	\text{Percentage change} =&  100*(6.6-5.4)/5.4 =22\%
	\end{align*}

	\vspace{0.2in}
	
\end{enumerate}

\noindent Q4. Answer the following short questions regarding our second half of studies into the Balance of Payments.

\begin{enumerate}[1)]
\item National savings for a given country can be subdivided into two items: domestic investment and foreign investment. Why do we treat the current account measure as our value of foreign investment? Describe an example to illustrate your point.

\vspace{0.2in}
Since $S = I + CA$ and $I$ represents the use of domestic savings, it must be the case that the remainder of $S$ is used for foreign investment. 
For example, if the US runs as CA deficit, this suggests a net inflow of funding wherein the US acts as a net borrower.
\vspace{0.2in}


\item If a country is labeled as a net borrower, what does this suggest about the value/sign of the current account balance?

\vspace{0.2in}
A net borrower has a current account deficit, meaning the current account balance is likely negative.
\vspace{0.2in}



\item Is the current account deficit a country may face necessarily a problem for that country? In what ways is it a sign of strength?

\vspace{0.2in}
Implies a vote of confidence by foreign investors. The US CA deficit since 1981 would be a strong nod towards the country's dependability.
\vspace{0.2in}

\item Under what circumstances can a current account deficit become a signal of weakness for a given economy? Elaborate with reference to examples from our Financial Crisis lecture.

\vspace{0.2in}
Excessive liquidity in a country can poison the chalice and lead to instability. Forbes and Warnock warn of this in their study of gross vs net capital flow assessments. 
\vspace{0.2in}

\item Suppose a given country maintains a highly unsustainable current account deficit and exhibits all the signs of a looming economic crash. 
Why might neighboring nations and those invested in said country be concerned about this matter? 

\vspace{0.2in}
They are likely to be more economically interdependent with the given crisis country (this is a prediction of the gravity model). 
Under these circumstances, they face greater risk of spillovers compared to more distant, less economically interwoven markets. 
\vspace{0.2in}


\end{enumerate}



\end{document}