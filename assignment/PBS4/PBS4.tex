%Preamble

\documentclass[12pt]{article}
\usepackage{amssymb}
\usepackage{amsfonts}
\usepackage{amsmath}
\usepackage[nohead]{geometry}
\usepackage{setspace}
\usepackage[bottom, hang, flushmargin]{footmisc}
\usepackage{indentfirst}
\usepackage{endnotes}
\usepackage{graphicx}
\usepackage{rotating}
\usepackage{natbib}
\usepackage{enumerate}
\usepackage{hyperref}
\setcounter{MaxMatrixCols}{30}
\newtheorem{theorem}{Theorem}
\newtheorem{acknowledgement}{Acknowledgement}
\newtheorem{algorithm}[theorem]{Algorithm}
\newtheorem{axiom}[theorem]{Axiom}
\newtheorem{case}[theorem]{Case}
\newtheorem{claim}[theorem]{Claim}
\newtheorem{conclusion}[theorem]{Conclusion}
\newtheorem{condition}[theorem]{Condition}
\newtheorem{conjecture}[theorem]{Conjecture}
\newtheorem{corollary}[theorem]{Corollary}
\newtheorem{criterion}[theorem]{Criterion}
\newtheorem{definition}[theorem]{Definition}
\newtheorem{example}[theorem]{Example}
\newtheorem{exercise}[theorem]{Exercise}
\newtheorem{lemma}[theorem]{Lemma}
\newtheorem{notation}[theorem]{Notation}
\newtheorem{problem}[theorem]{Problem}
\newtheorem{proposition}{Proposition}
\newtheorem{remark}[theorem]{Remark}
\newtheorem{solution}[theorem]{Solution}
\newtheorem{summary}[theorem]{Summary}
\newenvironment{proof}[1][Proof]{\noindent\textbf{#1.} }{\ \rule{0.5em}{0.5em}}
\geometry{left=1in,right=1in,top=1.00in,bottom=1.0in}

\begin{document}

\singlespacing

\noindent {EC 380: International Economic Issues \\ Instructor: P. Economides \\ Problem Set 4 \\ Fall 2022 \\ Due: 11:59 p.m. on Friday, Nov 25th}

\bigskip

\doublespacing



\section*{Setup}

\noindent 
Answers must be submitted online through Canvas by the stated deadline (see above).
Please prioritize posting your submission in PDF format.

\section*{Questions}

\noindent Q1. Answer the following short questions:

\begin{enumerate}[1)]
	
	\item Describe the current account, list and describe each of its three subcomponents. Describe one of these items in detail. 
	
	\vspace{2in}
	
	\item What type of activities does the capital account consist of?
	
	\vspace{2in}
	
	\newpage
	
	\item Regarding the financial account, what subcomponents does this measure include?
	
	\vspace{2.5in}
	
	\item How do the current, capital and financial accounts relate to one another when it comes to their numeric values?  
	
	\vspace{2.5in}
	
	\item Explain the reason why the balance of payments features a statistical discrepancy value.
	
	\vspace{1in}
	
\end{enumerate}

\newpage

\noindent Q2. Consider the following balance of payments for a given country.

\begin{table}[!h]
	\centering
	\begin{tabular}[t]{l l c }
		\hline
		&&\\
		ID & Description & Billions, USD \\
		&&\\
		\hline
		&&\\
		1. & Current Account Balance & 1000\\
		2. & Capital Account Balance & 200\\
		3. & Financial Account	& -\\
		3a. & Net acq of financial assets, excl financial der (increase/outflow (+))& 470\\
		3b. & Net inc of liabilities, excl financial der (increase/inflow (+))	& -890 \\
		3c. & Net change in financial derivatives & -200\\
		4. & Statistical Discrepancy & \\
		5. & Memoranda & \\
		5a. & Balance on current and capital accounts & \\
		5b. & Balance on financial account & \\
		\hline
	\end{tabular}
\end{table}




\begin{enumerate}[1)]
	\item In theory, what should the difference between items (5a.) and (5b.) be?
	
	\vspace{2.5in}
	
	\item Report the associated value of item (5a). Show your workings.
	
	\vspace{1in}
	
	
	\newpage
	
	\item Report the associated value of item (5b). Show your workings.
	
	\vspace{2in}
	
	\item What is the measure of statistical discrepancy measured as in this case (4.)?
	
	\vspace{2.5in}
	
	\item Would this country be considered a case of CA surplus or CA deficit?
	
	\vspace{2.5in}
	
\end{enumerate}

\newpage


\noindent Q3. Consider a case in which a given economy reports a GNP level of 5.4bn USD. 
The primary income net flows are work 1.2bn, while secondary income transfers are worth 0.2bn. 

\begin{enumerate}[1)]
	\item What is the implied level of GDP under these circumstances? Note that primary income is associated with income flows for compensating employees whereas secondary income is associated with transfers of income. 
	
	\vspace{3in}
	
	\item Demonstrate how the current account surplus is present in the measure of GNP, starting from the equation listed below.
	
	$$GNP = GDP + \text{Net Primary Income} + \text{Net Secondary Income}$$
	
	\vspace{1in}
	
	
	\newpage
	
	Using GDP, consider the fact that consumption (C) is equal to 1.5bn, investment (I) is 1.8bn and the government runs a balanced budget and collects 0.4bn in revenue (G). 
	
	\item What is the value of {\bf net} exports? Show your work.
	
	% 1.5 + 1.8 + 0.4 + NX = 4 = 1.5 + S + 0.4
	
	% C + I + G + X - M  = C + S + T = 4
	% C = 1.5, I = 1.8, G = 0.4, X=1, M=0.7
	% C = 1.5, S = 2.1, T = 0.4   
	
	\vspace{1.4in}
	
	\item What is the value of savings? Show your work.
	
	\vspace{2.5in}
	
	
	\item Suppose there is a shock to the economy, and the government is forced to run a major budget deficit 1.2bn.
	What does this imply about the tax rate for the country?
	
	\vspace{2.5in}
	
	\item Update your measure of GNP to reflect this change in the government budget balance. Show your workings. What is the percentage change in GNP?
	
	\vspace{2in}
	
\end{enumerate}

\noindent Q4. Answer the following short questions regarding our second half of studies into the Balance of Payments.

\begin{enumerate}[1)]
\item National savings for a given country can be subdivided into two items: domestic investment and foreign investment. Why do we treat the current account measure as our value of foreign investment? Describe an example to illustrate your point.

\vspace{2in}
\item If a country is labeled as a net borrower, what does this suggest about the value/sign of the current account balance?

\vspace{2in}


\newpage



\item Is the current account deficit a country may face necessarily a problem for that country? In what ways is it a sign of strength?

\vspace{2in}

\item Under what circumstances can a current account deficit become a signal of weakness for a given economy? Elaborate with reference to examples from our Financial Crisis lecture.

\vspace{2.5in}


\item Suppose a given country maintains a highly unsustainable current account deficit and exhibits all the signs of a looming economic crash. 
Why might neighboring nations and those invested in said country be concerned about this matter? 


\end{enumerate}



\end{document}