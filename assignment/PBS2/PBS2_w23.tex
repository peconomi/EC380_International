%Preamble

\documentclass[12pt]{article}
\usepackage{amssymb}
\usepackage{amsfonts}
\usepackage{amsmath}
\usepackage[nohead]{geometry}
\usepackage{setspace}
\usepackage[bottom, hang, flushmargin]{footmisc}
\usepackage{indentfirst}
\usepackage{endnotes}
\usepackage{graphicx}
\usepackage{rotating}
\usepackage{natbib}
\usepackage{enumerate}
\usepackage{hyperref}
\setcounter{MaxMatrixCols}{30}
\newtheorem{theorem}{Theorem}
\newtheorem{acknowledgement}{Acknowledgement}
\newtheorem{algorithm}[theorem]{Algorithm}
\newtheorem{axiom}[theorem]{Axiom}
\newtheorem{case}[theorem]{Case}
\newtheorem{claim}[theorem]{Claim}
\newtheorem{conclusion}[theorem]{Conclusion}
\newtheorem{condition}[theorem]{Condition}
\newtheorem{conjecture}[theorem]{Conjecture}
\newtheorem{corollary}[theorem]{Corollary}
\newtheorem{criterion}[theorem]{Criterion}
\newtheorem{definition}[theorem]{Definition}
\newtheorem{example}[theorem]{Example}
\newtheorem{exercise}[theorem]{Exercise}
\newtheorem{lemma}[theorem]{Lemma}
\newtheorem{notation}[theorem]{Notation}
\newtheorem{problem}[theorem]{Problem}
\newtheorem{proposition}{Proposition}
\newtheorem{remark}[theorem]{Remark}
\newtheorem{solution}[theorem]{Solution}
\newtheorem{summary}[theorem]{Summary}
\newenvironment{proof}[1][Proof]{\noindent\textbf{#1.} }{\ \rule{0.5em}{0.5em}}
\geometry{left=1in,right=1in,top=1.00in,bottom=1.0in}

\begin{document}

\singlespacing

\noindent {EC 380: International Economic Issues \\ Instructor: P. Economides \\ Problem Set 2 \\ Winter 2023 \\ Due: 11:59 p.m. on Friday, February 3rd}

\bigskip

\doublespacing



\section*{Setup}

\noindent 
Answers must be submitted online through Canvas by the stated deadline (see above).
Please prioritize posting your submission in PDF format.

\section*{Questions}

\noindent Q1. Answer the following short questions:

\begin{enumerate}[1)]
	
	\item How would you define {\bf Capital Abundance} under a Heckscher-Ohlin model setting?
	
	\vspace{1in}
	
	\item What does the capital-labor ratio of a specific pair of countries tell us about their patterns regarding trade specialization?
	
	\vspace{1in}
	
	\newpage
	
	\item Describe the key differences that distinguish the HO model from the Ricardian model. Why did the Ricardian model assumptions face concerns in latter years of use?
	
	\vspace{1.5in}
	
	\item How do outcomes play out for factor owners when trade opens? (E.g. What happens to labor earnings for the labor-abundant country?) Address capital and labor income in both settings, when a given country is abundant in either capital or labor. 
	
	\vspace{2in}
	
	\item Consider the Specific-Factor model. What are the implications for domestic labor income of switching from autarky to open trade, when a given country is land-abundant? Does this impact differ if the country is instead capital-abundant? 
	
	\vspace{1in}
	
\end{enumerate}

\newpage

\noindent Q2. Suppose we are considering a HO model setting, where countries have not yet opened up to trade.
Two goods are produced exclusively by domestic labor supplies, oil and fryers.
Suppose oil is labor-intensive in production, whereas constructing fryers is a capital intensive task.

\noindent Home and foreign maintain the following capital (K) and labor (L) endowments. 

\begin{table}[!h]
	\centering
	\begin{tabular}[t]{l c c }
		\hline
		&&\\
		Factor & L & K  \\
		&&\\
		\hline
		&&\\
		Home & 1250 & 500  \\
		Foreign & 1400 & 600 \\
		&&\\
		\hline
	\end{tabular}
\end{table}



\noindent Complete the following questions to highlight the two countries consumption and production equilibria. 

\begin{enumerate}[1)]

\item Which country has comparative advantage in producing oil?

\newpage

\item How will the Home PPF differ with respect to a Ricardian model? Draw an example of it, properly labeling your diagram.

\vspace{3.4in}

\item Explain the curvature of the PPF for countries in an HO setting. 

\bigskip

\bigskip

\bigskip

\bigskip

\newpage

\item Re-sketch the PPF for Home and introduce indifference curves. Which indifference curve is produced upon? Highlight this item and label the equilibrium point of production.


\vspace{4.0in}


\item Consider a shock to the economy where Home suddenly becomes endowed with additional capital and labor. Sketch what is expected to happen to the PPF and equilibrium production bundle.

\vspace{4in}

\end{enumerate}

\newpage


\noindent Q3. Suppose these countries transition into a free trade scenario.
Using the previous details about both countries, perform the following tasks.  

\noindent(i) Sketch a world relative price, CPC line, through your bundle such that it passes through the PPF (it will have no tangency to PPF). 

\noindent(ii) Using this CPC line, with its fixed slope, shift the line until it is perfectly tangent with the indifference curve. Update your sketch with this newly labeled CPC. 
	
\noindent(iii) Correctly label the equilibria points at which Home consumes and produces.
	
\noindent(iv) On your x-axis, and y-axis, indicate the area over which goods are either imports or exports, leveraging use of the differences in your consumption and production bundle. 
No values are necessary here, only ranges on your axes.

\noindent(v) Briefly comment on why countries do not completely specialize production. 


\newpage


\noindent Q4. Answer the following short questions:


\begin{enumerate}[1)]
	
	\item Describe the gravity model. What does it ignore?
	
	\vspace{1.6in}
	
	\item What inconsistencies between observed data and the HO model does the product cycle argument clear up?
	
	\vspace{1.6in}
	
	\item We covered inconsistencies/ambiguities of changes in labor outcomes, given greater openness to trade. Who does Autor et al (2013) suggest are the most vulnerable with respect to labor market outcomes in the US, following Chinese trade liberalization between 1990 and 2007? 
	
	\vspace{1.6in}
	
	\newpage
	
	\item In either high-skilled or low-skilled labor settings, how have cases of sudden increases in domestic labor supply via migration influenced wage rates? For full marks, cite the appropriate paper we considered with respect to this type of labor.
	
	\vspace{2in}
	
	\item Find a paper of interest related to trade and labor market outcomes. Cite the piece, including the journal it stems from, and describe its most interesting findings in a single paragraph.
	{\it Tips: If you're not accustomed to searching for papers, theres many ways to do so. I would recommend logging into the UO online library and searching 'Web of Science'. Follow the link through the library to this website and you'll be verified. From there, search the relevant key terms based on what we've covered in our class.
	For a smaller set of papers to consider, use Google Scholar and choose a relevant paper which best grips your attention. 
	Contact me if you're having any trouble choosing a piece.}
	
	\vspace{1in}
	
\end{enumerate}

\end{document}