%Preamble

\documentclass[12pt]{article}
\usepackage{amssymb}
\usepackage{amsfonts}
\usepackage{amsmath}
\usepackage[nohead]{geometry}
\usepackage{setspace}
\usepackage[bottom, hang, flushmargin]{footmisc}
\usepackage{indentfirst}
\usepackage{endnotes}
\usepackage{graphicx}
\usepackage{rotating}
\usepackage{natbib}
\usepackage{enumerate}
\usepackage{hyperref}
\setcounter{MaxMatrixCols}{30}
\newtheorem{theorem}{Theorem}
\newtheorem{acknowledgement}{Acknowledgement}
\newtheorem{algorithm}[theorem]{Algorithm}
\newtheorem{axiom}[theorem]{Axiom}
\newtheorem{case}[theorem]{Case}
\newtheorem{claim}[theorem]{Claim}
\newtheorem{conclusion}[theorem]{Conclusion}
\newtheorem{condition}[theorem]{Condition}
\newtheorem{conjecture}[theorem]{Conjecture}
\newtheorem{corollary}[theorem]{Corollary}
\newtheorem{criterion}[theorem]{Criterion}
\newtheorem{definition}[theorem]{Definition}
\newtheorem{example}[theorem]{Example}
\newtheorem{exercise}[theorem]{Exercise}
\newtheorem{lemma}[theorem]{Lemma}
\newtheorem{notation}[theorem]{Notation}
\newtheorem{problem}[theorem]{Problem}
\newtheorem{proposition}{Proposition}
\newtheorem{remark}[theorem]{Remark}
\newtheorem{solution}[theorem]{Solution}
\newtheorem{summary}[theorem]{Summary}
\newenvironment{proof}[1][Proof]{\noindent\textbf{#1.} }{\ \rule{0.5em}{0.5em}}
\geometry{left=1in,right=1in,top=1.00in,bottom=1.0in}

\begin{document}

\singlespacing

\noindent {EC 380: International Economic Issues \\ Instructor: P. Economides \\ Midterm Exam (Makeup) \\ Fall 2022 \\ Monday, 11/07/2022 11:00am-12:20pm}

\bigskip

\doublespacing



\section*{Setup}

\noindent 
This test will be comprised of 28 questions and must be attempted within 80 minutes of allocated time. 
\section*{Questions}

\noindent \underline{Question 1. Autarky Ricardian Outcomes}

\noindent Suppose we are considering a Ricardian model setting, where countries have not yet opened up to trade.
Two goods are produced exclusively by domestic labor supplies, Cookies and Cream. 
Home and Foreign maintain the following marginal productivities of labor (MPL) in producing products. 
The two countries labor pools $\bar{L}=25$ are equal in both countries.
Answer the following parts, worth 5pts each.

\begin{enumerate}[a)]
	
	\item Home has comparative advantage in producing...
	
	\vspace{1in}
	
	\item What is the maximum amount of Cream that Foreign can produce?
	
	\vspace{1in}
	
	\newpage
	
	\item In autarky, what is the home consumption bundle of {Cookies, Cream}?
	
	\vspace{2.8in}
	
	\item In autarky, what is the foreign consumption bundle of {Cookies, Cream}?  
	
	\vspace{2.8in}
	
	\item Upon opening up to trade, what will the production bundle of the Home country be?
	
	
\end{enumerate}

\newpage

\noindent \underline{Question 2: Short Theory Questions}

\noindent Answer the following short questions, each worth 5pts.
\begin{enumerate}[a)]
	\item Which one of the following trade models features a fixed opportunity cost of reallocating resources from one industry to another? Circle your answer.
	
	1. Gravity Model
	
	2. Product Cycle Model
	
	3. Ricardian Model
	
	4. Heckscher-Ohlin Model
	
	\bigskip
	
	\item True or false: A country with absolute advantage in producing both goods does not gain from trade.
	Explain your choice for full marks. 
	
	\vspace{2in}
	
	\item Which model uses underlying technological differences to explain patterns in trade, and why has this argument weakened relative in recent times?
	
	\vspace{2in}
	
	
	\newpage
	
	\item List three key features of the Heckscher-Ohlin model (e.g. number of goods). What mechanism determines the direction of trade flows?
	
	\vspace{2.5in}
	
	
	\item Suppose we are considering a HO model framework. If a country is described as being "capital-abundant", what does this entail about their factor endowments and export activity?
	
	\vspace{1.8in}
	
	\item If a country maintains status as a capital-abundant country, what happens to labor income upon opening up to trade and departing from autarky?
	
	\vspace{1.8in}
	
	\newpage
	
	\item The following description aligns which model? "Distance between countries and the size of individual economies are predictive of existing trade flows".
	
	1. Gravity 
	
	2. Heckscher-Ohlin
	
	3.Ricardian
	
	4. Product Cycle
	
	\bigskip
	
	\item In the Specific-Factors model, recall that labor is a variable factor input of production. 
	Suppose Home is capital-abundant, relative to Foreign. This implies Foreign is land-abundant. 
	Upon opening trade, what can we state about the change in labor income at Home?
	
	\vspace{1.8in}
	
\end{enumerate}

\newpage

%5+6+4

\noindent \underline{Question 3: General Equilibrium with Tariffs}

\noindent Suppose we are in an autarky scenario and considering the market for an imported good at Home. 
Use the following demand and supply functions for solving the various equilibrium scenarios.
\begin{align*}
\text{Demand:} \ \  P = & 415 - \frac{1}{7} Q_d\\
\text{Supply:}\ \  P = & 10  + \frac{1}{4} Q_s
\end{align*}

Ensure you show all of your calculations. Each part is worth 5pts. 

\begin{enumerate}[a)]
	\item Report the coordinates of the autarky equilibrium point, which represent the price and quantity the market operates at.
	
	\vspace{2in}
	
	\item Calculate the consumer and producer surplus values under autarky.  What is total welfare for the economy?
	
	\vspace{2.5in}
	
	\newpage
	
	\noindent Suppose Home opens up to free-trade, and becomes exposed to a world price, $P_w = 205$. 
	
	\item Calculate the free trade equilibrium values for quantities, imports and surplus values. 
	Report the change in welfare, relative to autarky. 
	
	\vspace{3in}
	
	\noindent Consider a case in which the government intervenes, setting a tariff rate of $t=15$. 
	
	\item Calculate the equilibria for quantity supplied, quantity demanded, imports and surpluses (consumer, producer, government). 
	Find the change in welfare, relative to free-trade. 
	
	\vspace{2in}
	
	\item Report the deadweight loss area value.
	
\end{enumerate}

\newpage

\noindent \underline{Question 4: Short Inference Tables}

\noindent Tariffs on final and input goods are 23\% and 56\%, respectively. Complete the entries in the table below, expressing the effective rate of protection in each case.
When calculating the effect of an input tariff, assume the final good tariff remains in place.
5 pts per question. 

\begin{enumerate}[a)]
	
	\item Calculate the effective rate of protection under the final good tariff rate only.
	
		\vspace{2in}
	
	\item Calculate the effective rate of protection under a scenario in which both tariffs are in effect.
	
	\vspace{2in}

	\begin{table}[!h]
		\centering
		\begin{tabular}[t]{l c c c}
			\hline
			&&&\\
			Variable & No Tariff & + Tariff on Final Good & + Tariff on Input Good \\
			&&&\\
			\hline
			&&&\\
			Price of Domestic Final Good & 3200 & & \\
			Value of Imported Inputs & 250 & & \\
			Domestic Value-Added &	2950	&&\\
			&&&\\
			Effective Rate of Protection, \% &	0	&& \\
			&&&\\
			\hline
		\end{tabular}
	\end{table}

	\newpage
	
Suppose the UK and France maintain the following capital and labor endowments.
\begin{table}[!h]
	\centering
	\begin{tabular}[t]{l c c }
		\hline
		&&\\
		 & Capital & Labor \\
		&&\\
		\hline
		&&\\
		UK & 455 & 480 \\
		&&\\
		France & 540 & 650 \\
		\hline
	\end{tabular}
\end{table}

\item Which country has comparative advantage in producing the labor-intensive good?

\vspace{2in}
 
\item How would matters change in terms of comparative advantages if suddenly migrants entered France, causing labor to jump up by an additional 200 units?

\end{enumerate}

\newpage

\noindent \underline{Question 5: Short Tariff \& Trade Policy Questions}

\begin{enumerate}[a)]

\item Explain with visual aid why producers prefer the quota system over the tariff system.

\vspace{4in}

\item If a country shares zero tariff rates with fellow members of this trade agreement and but each member has unique third-party tariff schedules, they are part of a...

\vspace{2in}

\newpage

\item When a large country imposes a tariff adjustment, do the resulting consequences play out differently to the small country scenario? Explain.


\vspace{1.5in}

\item The UK-EU "Brexit" deal was effective as of January 2020. Which of these international issues remains a key contention in the framework of the new treaty?

\vspace{1.5in}

\item Define an act of `trade diversion' and use an example.

\vspace{1.5in}

\item Complete the following statement: ``I consider this test to be (blank) in difficulty." 
Choose from: Easy, Moderate, Difficult, Outrageous.

\end{enumerate}

\end{document}